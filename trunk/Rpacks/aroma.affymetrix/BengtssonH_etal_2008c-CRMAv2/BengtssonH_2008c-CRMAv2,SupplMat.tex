\NeedsTeXFormat{LaTeX2e}[1995/12/01]
\documentclass[10pt,a4paper]{article}    

% Load packages
\usepackage{url}  % Formatting web addresses  
\usepackage{ifthen}  % Conditional 
\usepackage{multicol}   %Columns
\usepackage[utf8]{inputenc} %unicode support
%\usepackage[applemac]{inputenc} %applemac support if unicode package fails
%\usepackage[latin1]{inputenc} %UNIX support if unicode package fails
\urlstyle{rm}
\usepackage{fancyvrb}

% To refer to labels to the manuscript
\usepackage{xr} 
\externaldocument[CRMAv2:]{BengtssonH_2008c-CRMAv2}
 
 
\setlength{\topmargin}{-2.0cm}
\setlength{\textheight}{25.5cm}
\setlength{\oddsidemargin}{0cm} 
\setlength{\textwidth}{16.5cm}
\setlength{\columnsep}{0.6cm}

% - - - - - - - - - - - - - - - - - - - - - - - - - - - - - - - - 
% Controlling page disposition of Tables and Figures
% - - - - - - - - - - - - - - - - - - - - - - - - - - - - - - - - 
\renewcommand{\topfraction}{1.0}	% max fraction of floats at top
\renewcommand{\bottomfraction}{1.0}	% max fraction of floats at bottom
\renewcommand{\textfraction}{0.07}
 
% - - - - - - - - - - - - - - - - - - - - - - - - - - - - - - - - 
% User-specific packages
% - - - - - - - - - - - - - - - - - - - - - - - - - - - - - - - - 
\usepackage{amsmath} 
\usepackage{xspace}
\usepackage{bm} \bm{}
\usepackage{graphicx}


% - - - - - - - - - - - - - - - - - - - - - - - - - - - - - - - - 
% Graphics settings
% - - - - - - - - - - - - - - - - - - - - - - - - - - - - - - - - 
% The search path (within curly brackets and separated by commas)
% where to find graphics files
\graphicspath{{figures/col/},{figures/}}
% Make dvips deal with gzip'ed EPS figures.
\DeclareGraphicsRule{.eps.gz}{eps}{.eps.bb}{`gunzip -c #1}


% - - - - - - - - - - - - - - - - - - - - - - - - - - - - - - - - 
% User-specific macros
% - - - - - - - - - - - - - - - - - - - - - - - - - - - - - - - - 
\newcommand{\GWSFive}{GWS5\xspace}
\newcommand{\GWSSix}{GWS6\xspace}
\newcommand{\GWSFivef}{GenomeWideSNP\_5\xspace}
\newcommand{\GWSSixf}{GenomeWideSNP\_6\xspace}

\newcommand{\PMA}{\ensuremath{\textnormal{PM}_\textnormal{A}}\xspace}
\newcommand{\PMB}{\ensuremath{\textnormal{PM}_\textnormal{B}}\xspace}
\newcommand{\TP}{\ensuremath{\textnormal{TP}}\xspace}
% MathOperator!
\newcommand{\MAD}{\ensuremath{\textnormal{MAD}}\xspace}

\newcommand{\chrX}{ChrX\xspace}
\newcommand{\chrY}{ChrY\xspace}
\newcommand{\NspI}{\emph{Nsp}I\xspace}
\newcommand{\StyI}{\emph{Sty}I\xspace}
\newcommand{\Nsp}{\ensuremath{\textnormal{Nsp}}\xspace}
\newcommand{\Sty}{\ensuremath{\textnormal{Sty}}\xspace}

\newcommand{\filename}[1]{\textit{#1}\xspace}
%\newcommand{\url}[1]{\href{#1}{#1}\xspace}
\newcommand{\pkg}[1]{\textit{#1}\xspace}
\newcommand{\args}[1]{\textit{#1}\xspace}
\newcommand{\FL}{\textnormal{FL}\xspace}
\newcommand{\GC}{\textnormal{GC}\xspace}
\newcommand{\CN}{CN\xspace}
\newcommand{\kb}{\textnormal{kb}\xspace}
\newcommand{\PM}{\textnormal{PM}\xspace}
\newcommand{\MM}{\textnormal{MM}\xspace}
\newcommand{\bx}{\mathbf{x}\xspace}
\newcommand{\by}{\mathbf{y}\xspace}
\newcommand{\bz}{\mathbf{z}\xspace}
\newcommand{\ba}{\mathbf{a}\xspace}
\newcommand{\bA}{\mathbf{A}\xspace}
\newcommand{\bB}{\mathbf{B}\xspace}
\newcommand{\bC}{\mathbf{C}\xspace}
\newcommand{\bS}{\mathbf{S}\xspace}
\newcommand{\bU}{\mathbf{U}\xspace}
\newcommand{\bV}{\mathbf{V}\xspace}
\newcommand{\bW}{\mathbf{W}\xspace}
\newcommand{\beps}{\bm{\varepsilon}\xspace}
\newcommand{\bnu}{\bm{\nu}\xspace}
\renewcommand{\Re}{\mathbb{R}\xspace}
\newcommand{\II}{\mathbb{I}\xspace}
\DeclareMathOperator{\median}{\textnormal{median}}
\DeclareMathOperator{\mad}{\textnormal{mad}}

\newcommand{\updated}[3][red]{{{\color{#1}\textsl{\textbf{#2}}}\endnote{#3 \textsl{#2}}}\xspace}
\renewcommand{\updated}[3][red]{#2\xspace} 

\usepackage{natbib}
%\newcommand{\citet}[1]{\cite{#1}} 
%\newcommand{\citep}[1]{\cite{#1}} 


% Begin ...
\begin{document}


\title{Supplementary materials for CRMA v2}
\author{Henrik Bengtsson et al.}
\maketitle
\tableofcontents

\clearpage
\section{Annotation data files}
Although the features on the arrays never change, their annotations might get updated as the Human Genome databases get updated.  In this study, we have used the latest annotation data available on the Affymetrix website as of August 2008.
We will list all annotation files needed by CRMA~v2 for analyzing data for the \GWSFivef (GWS5) and the \GWSSixf (GWS6) chip types.  First, we specify all relevant files available from Affymetrix (Tables~\ref{tblGWSFiveAffymetrix}~\&~\ref{tblGWSSixAffymetrix}).  Then we specify the files compiled from the Affymetrix files for use in aroma.affymetrix (Tables~\ref{tblGWSFiveAromaAffymetrix}~\&~\ref{tblGWSSixAromaAffymetrix}).

\subsection{Affymetrix annotation data}

\begin{table}[hp]
\begin{center}
\begin{tabular}{lp{0.7\textwidth}}
\hline
File name    & GenomeWideSNP\_5,r2.cdf \\
File size    & 239,670,727 bytes \\
MD5 checksum & e062cb42a392dd74721d5b7cf9a286f1 \\
Notes        & Affymetrix default chip definition file (CDF). \\
\hline
File name    & GenomeWideSNP\_5,Full,r2.cdf \\
File size    & 261,578,689 bytes \\
MD5 checksum & 79f7a8353b4978dedbeff05a7897ff6e \\
Notes        & Affymetrix full chip definition file (CDF). \\
\hline
File name    & GenomeWideSNP\_5.CN\_probe\_tab \\
File size    & 43,252,969 bytes \\
MD5 checksum & e661544a27163d1242ca690a486cb6b5 \\
Notes        & Affymetrix file probe sequence file for CN probes. \\
\hline
File name    & GenomeWideSNP\_5.probe\_tab \\
File size    & 233,706,497 bytes \\
MD5 checksum & 69b66720591fde9333acf5eb4d1b3e68 \\
Notes        & Affymetrix file probe sequence file for SNPs. \\
\hline
File name    & GenomeWideSNP\_5.cn.na26.annot.csv \\
File size    & 180,849,309 bytes \\
MD5 checksum & ec4f6cb4b482923d73d07f1b07faefe4 \\
Notes        & Affymetrix NetAffx v26 annotation file for CN probes. \\
\hline
File name    & GenomeWideSNP\_5.na26.annot.csv \\
File size    & 755,337,946 bytes \\
MD5 checksum & af59235b6fccada7f871257149a89215 \\
Notes        & Affymetrix NetAffx v26 annotation file for SNPs. \\
\hline
\end{tabular}
\end{center}
\caption{Details of all Affymetrix specific GenomeWideSNP\_5 annotation files used in the study.}
\label{tblGWSFiveAffymetrix}
\end{table}


\begin{table}[hp]
\begin{center}
\begin{tabular}{lp{0.7\textwidth}}
\hline
File name    & GenomeWideSNP\_6.cdf \\
File size    & 484,489,553 bytes \\
MD5 checksum & 223f3cd9141404b2a926a40cf47d6f1a \\
Notes        & Affymetrix default chip definition file (CDF). \\
\hline
File name    & GenomeWideSNP\_6.Full.cdf \\
File size    & 493,291,745 bytes \\
MD5 checksum & 3fbe0f6e7c8a346105238a3f3d10d4ec \\
Notes        & Affymetrix full chip definition file (CDF). \\
\hline
File name    & GenomeWideSNP\_6.CN\_probe\_tab \\
File size    & 96,968,290 bytes \\
MD5 checksum & 3dc2d3178f5eafdbea9c8b6eca88a89c \\
Notes        & Affymetrix file probe sequence file for CN probes. \\
\hline
File name    & GenomeWideSNP\_6.probe\_tab \\
File size    & 341,479,928 bytes \\
MD5 checksum & 2037c033c09fd8f7c06bd042a77aef15 \\
Notes        & Affymetrix file probe sequence file for SNPs. \\
\hline
File name    & GenomeWideSNP\_6.cn.na26.annot.csv \\
File size    & 482,222,873 bytes \\
MD5 checksum & 948eb406774aa5097590debd0d667a22 \\
Notes        & Affymetrix NetAffx v26 annotation file for CN probes. \\
\hline
File name    & GenomeWideSNP\_6.na26.annot.csv \\
File size    & 1,628,608,540 bytes \\
MD5 checksum & 323f9afa0c180c146260b5eb689d0bd2 \\
Notes        & Affymetrix NetAffx v26 annotation file for SNPs. \\
\hline
\end{tabular}
\end{center}
\caption{Details of all Affymetrix specific GenomeWideSNP\_6 annotation files used in the study.}
\label{tblGWSSixAffymetrix}
\end{table}


\clearpage
\subsection{Imported annotation data}
Several of Affymetrix annotation files use file formats that are not intended to be used directly in a computational system, but rather be imported once.  The \pkg{aroma.affymetrix} framework utilizes its own binary files that are more compact and faster to access.  The contents of these are imported from the above Affymetrix annotation data files.  Further details on the source files used for each compiled file is given in the file footer of each file.  To download these and for further details, see the aroma.affymetrix webpage.

\begin{table}[hp]
\begin{center}
\begin{tabular}{lp{0.7\textwidth}}
\hline
File name    & GenomeWideSNP\_5,HB20080710.acs \\
File size    & 121,981,027 bytes \\
MD5 checksum & bd0da64b09ea164082e066798774a3c5 \\
Notes        & Aroma.affymetrix probe sequence data from above Affymetrix sequence files. \\
\hline
File name    & GenomeWideSNP\_5,Full,r2,na26,HB20080822.ufl \\
File size    & 3,684,511 bytes \\
MD5 checksum & a533ac3ba64f36902d13bed8fe2a9a5b \\
Notes        & Aroma.affymetrix fragment length data mapping to the full CDF.  Compiled from above Affymetrix NetAffx files. \\
\hline
File name    & GenomeWideSNP\_5,Full,r2,na26,HB20080822.ugp \\
File size    & 4,605,439 bytes \\
MD5 checksum & 9827c9fad08144d6e590b87984350a26 \\
Notes        & Aroma.affymetrix genome location data mapping to the full CDF.  Compiled from above Affymetrix NetAffx files. \\
\hline
File name    & GenomeWideSNP\_5,r2,na26,HB20080822.ufl \\
File size    & 3,445,230 bytes \\
MD5 checksum & 566812ef309486baf0a8496bb8a35ef5 \\
Notes        & Aroma.affymetrix fragment length data mapping to the default CDF.  Compiled from above Affymetrix NetAffx files. \\
\hline
File name    & GenomeWideSNP\_5,r2,na26,HB20080822.ugp \\
File size    & 4,306,339 bytes \\
MD5 checksum & 9330fdcb30b3a3b4c13cbdeb8de4ffb5 \\
Notes        & Aroma.affymetrix genome location data mapping to the default CDF.  Compiled from above Affymetrix NetAffx files. \\
\hline
\end{tabular}
\end{center}
\caption{Details of all aroma.affymetrix specific GenomeWideSNP\_5 annotation files used in the study.}
\label{tblGWSFiveAromaAffymetrix}
\end{table}



\begin{table}[hp]
\begin{center}
\begin{tabular}{lp{0.7\textwidth}}
\hline
File name    & GenomeWideSNP\_6,HB20080710.acs \\
File size    & 179,217,531 bytes \\
MD5 checksum & f04f081e0a1900653d957a8f320744c0 \\
Notes        & Aroma.affymetrix probe sequence data from above Affymetrix sequence files. \\
\hline
File name    & GenomeWideSNP\_6,Full,na26,HB20080722.ufl \\
File size    & 7,526,454 bytes \\
MD5 checksum & 6f11e9bd3a7a0cb060d5fcf671b0776a \\
Notes        & Aroma.affymetrix fragment length data mapping to the full CDF.  Compiled from above Affymetrix NetAffx files. \\
\hline
File name    & GenomeWideSNP\_6,Full,na26,HB20080821.ugp \\
File size    & 9,407,937 bytes \\
MD5 checksum & 5a7bef30a458cb238ae2167aa41f5bd6 \\
Notes        & Aroma.affymetrix genome location data mapping to the full CDF.  Compiled from above Affymetrix NetAffx files. \\
\hline
File name    & GenomeWideSNP\_6,na26,HB20080821.ufl \\
File size    & 7,425,058 bytes \\
MD5 checksum & 522b89d875f39832f5423e78cffba8c8 \\
Notes        & Aroma.affymetrix fragment length data mapping to the full CDF.  Compiled from above Affymetrix NetAffx files. \\
\hline
File name    & GenomeWideSNP\_6,na26,HB20080821.ugp \\
File size    & 9,281,127 bytes \\
MD5 checksum & ad63ef009b44f1274f6e2a35cb951dbc \\
Notes        & Aroma.affymetrix genome location data mapping to the default CDF.  Compiled from above Affymetrix NetAffx files. \\
\hline
\end{tabular}
\end{center}
\caption{Details of all aroma.affymetrix specific GenomeWideSNP\_6 annotation files used in the study.}
\label{tblGWSSixAromaAffymetrix}
\end{table}



%%%%%%%%%%%%%%%%%%%%%%%%%%%%%%%%%%%%%%%%%%%%%%%%%%%%%%%%%%%%%%%%%%%%%%%%%%%
% CDF files
%%%%%%%%%%%%%%%%%%%%%%%%%%%%%%%%%%%%%%%%%%%%%%%%%%%%%%%%%%%%%%%%%%%%%%%%%%%
\clearpage
\section{Summary of annotation data}

This section provides a summary of the relevant annotation data available for the GWS5 and the GWS6 chip types.  We separate between the annotation data available in the Chip Definition Files (CDFs) and the NetAffx files, because the former are less likely to change over time whereas the latter gets updates when the genome annotations get updated.  Affymetrix' NetAffx annotation data are updated several times a year.

\subsection{Affymetrix CDF data}

Affymetrix provides one ``default'' and one ``full'' CDFs for each of the GWS5 and GWS6 chip types.  See Table~\ref{tblGWSSixAffymetrix} for which these files are.
The default CDF contains a subset of the full CDF where certain SNP units have been filter out due to homology to other regions and poor performance (private communication with Affymetrix).
Tables~\ref{tblCdfUnits}~\&~\ref{tblCdfProbes} provides a summary of these CDFs.
All summaries presented here and in the main paper refer to the full CDFs, unless stated otherwise.

\subsubsection{Unit annotations}

\begin{table}[hp]
\begin{center}
\begin{tabular}{|l|rr||rr|}
\hline
                        &  \GWSFive &  \GWSFive &   \GWSSix  &   \GWSSix \\
                        &  default  &  full        & default &   full    \\
\hline
\hline
UNIT TYPES              &           &           &            &           \\
CN units                &  417,269  &  417,269  &   945,826  &   945,826 \\
SNPs                    &  440,794  &  500,568  &   906,600  &   931,946 \\
Subtotal                &  858,063  &  917,847  & 1,852,426  & 1,877,772 \\
AFFX-SNPs               &    3,022  &    3,022  &     3,022  &     3,022 \\
Subtotal                &  861,085  &  920,859  & 1,855,448  & 1,880,794 \\
Others                  &       24  &       69  &       621  &       621 \\
Total                   &  861,109  &  920,928  & 1,856,069  & 1,881,415 \\
\hline									            						             
EXCLUDED FROM FULL      &           &           &            &           \\
CN units                &        0  &        -  &         0  &         - \\
SNPs                    &   59,744  &        -  &    25,346  &         - \\
AFFX-SNPs               &        0  &        -  &         0  &         - \\
Others                  &       45  &        -  &         0  &         - \\
\hline									            						             
CN UNIT STRANDNESS 		  &	  				&	  				&	 				   &	 		     \\
Sense only              &        0  &        0  &         0  &         0 \\
Antisense only          &  417,269  &  417,269  &   945,826  &   945,826 \\
Opposite strands        &        0  &        0  &         0  &         0 \\
Both strands            &        0  &        0  &         0  &         0 \\
\hline									            						             
SNP STRANDNESS					&	  				&	  				&	 				   &	 		     \\
Sense only              &  234,449  &  260,266  &   477,538 &    491,830 \\
Antisense only          &  174,549  &  194,126  &   429,062  &   440,116 \\
Opposite strands        &   31,796  &   46,176  &         0  &         0 \\
Both strands            &        0  &        0  &         0  &         0 \\
\hline									            						             
AFFX-SNP STRANDNESS 		&	  				&	  				&	 				   &	 		     \\
Sense only              &      145  &      145  &       145  &       145 \\
Antisense only          &      129  &      129  &       129  &       129 \\
Opposite strands        &        0  &        0  &         0  &         0 \\
Both strands            &    2,748  &    2,748  &     2,748  &     2,748 \\
\hline									            						             
SNP ALIGNMENT	           &	  			 &           &	 		      &	 		      \\
Aligned allele pairs     &  285,984  &  308,169  &   906,600  &   931,946 \\
Non-aligned allele pairs &  154,810  &  192,399  &         0  &         0 \\
\hline									            						             
AFFX-SNP ALIGNMENT	     &	  			 &	  				&	 		       &	 		      \\
Aligned allele pairs     &    3,022  &    3,022  &    3,022  &     3,022  \\
Non-aligned allele pairs &        0  &        0  &        0  &         0  \\
\hline									            						             
\end{tabular}
\end{center}
\caption{Summary of the unit annotation available in the CDF files.  All values are in counts.}  %% Summary generated 2008-08-22.
\label{tblCdfUnits}
\end{table}


\clearpage
\subsubsection{Probe annotations}
\begin{table}[hp]
\begin{center}
\begin{tabular}{|l|rr||rr|}
\hline
                        &  \GWSFive &  \GWSFive &   \GWSSix  &   \GWSSix \\
                        &  default  &  full        & default &   full    \\
\hline
\hline

PROBES                  &           &           &            &           \\
CN probes               &   417,269 &   417,269 &    945,826 &   945,826 \\
SNP probes              & 3,526,352 & 4,004,544 &  5,660,710 & 5,833,210 \\
Subtotal                & 3,943,621 & 4,421,813 &  6,606,536 & 6,779,036 \\
AFFX-SNP probes         &    81,504 &    81,504 &     81,504 &    81,504 \\
Subtotal                & 4,025,125 & 4,503,317 &  6,688,040 & 6,860,540 \\
Other probes            &   178,055 &   188,239 &     32,420 &    32,420 \\
Excluded from full      &   488,376 &         0 &    172,500 &         0 \\
Total                   & 4,691,556 & 4,691,556 &  6,892,960 & 6,892,960 \\
\hline									            						             
EXCLUDED FROM FULL      &           &           &            &           \\
CN probes               &        0  &        -  &         0  &         - \\
SNP probes              &  478,192  &        -  &   172,500  &         - \\
AFFX-SNP probes         &        0  &        -  &         0  &         - \\
Other probes            &   10,184  &        -  &         0  &         - \\
\hline									            						             
CN UNIT PROBES           &	  	     &	  			 &	 				 &	 				 \\
CN units with 1 probe    & 417,269  &  417,269  &   945,826 &   945,826 \\
\hline
SNP PROBE PAIRS         &	  				&	  				&	 				   &	 				 \\
SNPs with 3 pairs       &        0  &        0  &    796,045 &   811,179 \\
SNPs with 4 pairs       &  440,794  &  500,568  &    110,555 &   120,767 \\
\hline
AFFX-SNP PROBE PAIRS     &	         &	  			 &	 				  &	 		      \\
SNPs with 12 pairs       &    2,461  &    2,461  &     2,461  &     2,461 \\
SNPs with 20 pairs       &      561  &      561  &       561  &       561 \\
\hline
\end{tabular}
\end{center}
\caption{Summary on the probe annotation available in the CDF files.  All values are in counts.}    %% Summary generated 2008-08-22.
\label{tblCdfProbes}
\end{table}



\subsection{Affymetrix NetAffx data}

Affymetrix makes so called NetAffx CSV files available for download.  These files contain annotation data exported from their NetAffx data base.  For each chip type there exists one or more NetAffx CSV files, e.g. for the GWS chip types there is one for the CN units and one for all other units on the chip.  See Tables~\ref{tblGWSFiveAffymetrix}~\&~\ref{tblGWSSixAffymetrix} for which these files are.
The NetAffx data base is updated frequently and the following information is likely to get slightly outdated over time. 

\subsubsection{Genome positions}
In Table~\ref{tblGenomicLocation} we summarize how many SNPs and CN loci have known annotations according to NetAffx.  
The genome location per chromosomes is summarized in Table~\ref{tblGenomicLocationByChromosome} for each of the default and the full CDF.

We do not know why the location is unknown for some loci, but from our investigation we believe it is mainly because such loci map to multiple positions in the genome.  For instance, in NetAffx release 26, there is no genomic location reported for SNP\_A-4228947 (on GWS5, GWS6 and Mapping250K\_Nsp), and according to the NCBI SNP data base it (rs11261805) maps to the two locations 41,240,208 and 43,493,496 on Chr9.

Furthermore, for GWS5 the NetAffx annotation files currently available only contains data on the units in the default CDF.  This is confirmed by comparing the names of the units with known locations in the default and the full GWS5.

\begin{table}[hp]
\begin{center}
\begin{tabular}{|l||rr||rr|}
\hline
     & \multicolumn{2}{c||}{\GWSFive} & \multicolumn{2}{c|}{\GWSSix} \\
     & default & full* & default & full \\
     & \#loci & \#loci & \#loci & \#loci \\
\hline
\hline
SNPs with known locations        & 440,094 &  440,094  &   905,386 &   929,967 \\
SNPs with unknown locations      &     700 &   60,474  &     1,214 &     1,979 \\
\hline
CN probes with known locations   & 312,384 &  312,384  &   945,806 &   945,806 \\
CN probes with unknown locations & 104,885 &  104,885  &        20 &        20 \\
\hline
AFFX-SNPs with known locations   &   3,012 &    3,012  &     3,012 &     3,012 \\
AFFX-SNPs with unknown locations &      10 &       10  &        10 &        10 \\
\hline
Total with known locations       & \textbf{755,490} & \textbf{755,490} & \textbf{1,854,204}  & \textbf{1,878,785} \\
Total with unknown locations     & 165,369 &   92,121  &    1,244  &     2,009 \\
\hline
Total                            & 861,085 &  920,859  & 1,855,448 & 1,880,794 \\
\hline
\end{tabular}
\end{center}
\caption{Summary of genomic location data that is available in the NetAffx files of contents in GWS5 and GWS6 with respect to unit and probe class and availability of annotation data.  The effective sets of units available for CN analysis are emphasized in bold.  *See text for why the default and the full GWS5 are identical.  NetAffx v26 was used for this summary.}
\label{tblGenomicLocation}
\end{table}

\begin{table}[hp]
\begin{center}
\begin{tabular}{|r|r||rr||rr|}
\hline
     & & \multicolumn{2}{c||}{\GWSFive} & \multicolumn{2}{c|}{\GWSSix} \\
     & & default & full* & default & full \\
chromosome & seq. length (Mbs) & \#loci & \#loci & \#loci & \#loci \\
\hline
\hline
1  & 245.2   & 58,548  &  58,548  &  144,499  &  146,401  \\
2  & 243.3   & 63,380  &  63,380  &  151,902  &  153,663  \\
3  & 199.4   & 53,120  &  53,120  &  126,337  &  127,766  \\
4  & 191.6   & 50,594  &  50,594  &  118,933  &  120,296  \\
5  & 181.0   & 48,661  &  48,661  &  114,333  &  115,672  \\
6  & 170.7   & 47,329  &  47,329  &  111,440  &  112,825  \\
7  & 158.4   & 39,325  &  39,325  &   99,818  &  100,996  \\
8  & 145.9   & 41,134  &  41,134  &   97,040  &   98,277  \\
9  & 134.5   & 31,563  &  31,563  &   81,036  &   82,168  \\
10 & 135.5   & 39,140  &  39,140  &   92,331  &   93,592  \\
\hline
11 & 135.0   & 37,948  &  37,948  &   88,295  &   89,525  \\
12 & 133.5   & 36,422  &  36,422  &   86,209  &   87,321  \\
13 & 114.2   & 28,119  &  28,119  &   65,310  &   66,067  \\
14 & 105.3   & 23,621  &  23,621  &   56,339  &   57,103  \\
15 & 100.1   & 20,968  &  20,968  &   52,810  &   53,556  \\
16 &  90.0   & 20,718  &  20,718  &   53,329  &   54,182  \\
17 &  81.7   & 17,411  &  17,411  &   46,024  &   46,632  \\
18 &  77.8   & 21,870  &  21,870  &   51,510  &   52,093  \\
19 &  63.8   & 10,631  &  10,631  &   29,855  &   30,299  \\
20 &  63.6   & 17,911  &  17,911  &   43,052  &   43,628  \\
\hline
21 &  47.0   & 10,412  &  10,412  &   24,787  &   25,111  \\
22 &  49.5   &  9,346  &   9,346  &   24,000  &   24,484  \\
X  & 152.6   & 26,373  &  26,373  &   86,064  &   87,198  \\
Y  &  51.0   &    946  &     946  &    8,841  &    9,485  \\
\hline
Mitochondrial  &  16.6kb &      -  &       -  &      110  &      445  \\
\hline
total  &     & 755,490 &  755,490 & 1,854,204 & 1,878,785 \\
\hline
\end{tabular}
\end{center}
\caption{Distribution of loci by chromosome for the different CDFs of GWS5 and GWS6.  *See text for why the default and the full GWS5 are identical.  NetAffx v26 was used for this summary.}
\label{tblGenomicLocationByChromosome}
%% Chromosome (sequence) lengths: http://www.ornl.gov/sci/techresources/Human_Genome/posters/chromosome/faqs.shtml
%% Summaries generated 2008-08-22 using GenomeWideSNP_5,r2,na26,HB20080822 and GenomeWideSNP_5,Full,r2,na26,HB20080822.
\end{table}


\clearpage
\subsubsection{Restriction enzymes and PCR fragment lengths}
\begin{table}[hp]
\begin{center}
\begin{tabular}{|l||rr||rr|}
\hline
     & \multicolumn{2}{c||}{\GWSFive} & \multicolumn{2}{c|}{\GWSSix} \\
     & default & full* & default & full \\
     & \#loci & \#loci & \#loci & \#loci \\
\hline
\hline
SNPs on NspI only        & 116,979 &  116,979  &   240,001 &   246,080 \\
SNPs on StyI only        &  74,135 &   74,135  &   154,884 &   160,899 \\
SNPs on both             & 248,980 &  248,980  &   510,330 &   522,472 \\
\hline
SNPs with known lengths  & 440,094 &  440,094  &   905,215 &   929,451 \\
SNPs with unknown lengths&     700 &   60,474  &     1,385 &     2,495 \\
\hline
\hline
CN probes on NspI only   & 140,099 &  140,099  &   451,191 &   451,191 \\
CN probes on StyI only   &   1,208 &    1,208  &         0 &         0 \\
CN probes on both        & 171,077 &  171,077  &   494,615 &   494,615 \\
\hline
CN probes with known lengths   & 312,384 &  312,384  &   945,806 &   945,806 \\
CN probes with unknown lengths & 104,885 &  104,885  &        20 &        20 \\
\hline
\hline
AFFX-SNPs with known lengths   &       0 &        0  &         0 &         0 \\
AFFX-SNPs with unknown lengths &   3,022 &    3,022  &     3,022 &     3,022 \\
\hline
\hline
Total with known lengths       & 752,478 &  752,478  & 1,851,021 & 1,875,257 \\
Total with unknown lengths     & 108,607 &  168,381  &     4,427 &     5,537 \\
\hline
Total                          & 861,085 &  920,859  & 1,855,448 & 1,880,794 \\
\hline
\end{tabular}
\end{center}
\caption{Summary of fragment-length data that is available in the NetAffx files of contents in GWS5 and GWS6 with respect to unit and probe class and availability of annotation data.  *See text for why the default and the full GWS5 are identical.  NetAffx v26 was used for this summary.}
\label{tblFragmentLengths}
\end{table}





\clearpage
\section{Multi-enzyme digestion}
\label{secMultiEnzymeDigestion}
For the 100K as well as the 500K SNP-only assays, DNA is prepared in two parallel processes, each digesting the DNA using a unique restriction enzyme, amplifying the fragments by PCR, and hybridizing the products to separate arrays.  In the GWS assays, which similarly to 500K uses enzymes \NspI and \StyI, the two mixes of PCR products are no longer hybridized to separate arrays but instead hybridized in aliquot to the same array~\citep{Affymetrix_2007f,Affymetrix_2007g}.
Consequently, SNP target DNA of PCR products originating from different digestions may hybridize to the same probe, which is something that has to be taken into account when, for instance, fitting the fragment-length normalization.
For CN probes the situation is somewhat different.  Affymetrix selected the CN probes from a large pool of CN probes based on their performance on copy numbers (private communication).  This pilot study was conducted on a specially designed in-house chip set containing probes that are known to be on an \NspI fragment.  For this reason, some of the selected probes are exclusively on \NspI fragments, some are by chance both on \NspI and \StyI fragments, but none are exclusively on \StyI fragments.  Note, when annotation for the human genome is updated, some of the probes might by chance be re-annotated to become \StyI-only probes.
We have found that it is important that the preprocessing models these differences, otherwise there is a substantial risk for getting systematic biases between SNPs and CN probes due to enzymatic mixing imbalances.
%% [We will later see that the hybridization of two PCR products on the same array requires a special model in order to control for confounded effects due to mixing inbalances.]
See Table~\ref{tblFragmentLengths} for details on fragment-length information for the two chip types and the two enzymes.

%% Note that there is no enzyme annotation data available for a large set of the GWS5 SNP units.  Instead, we use \GWSSix annotation data for the common SNPs and for CN probes for which annotation data otherwise was missing.  For further details and updates on this problem, see The aroma.affymetrix website [\url{http://www.braju.com/R/aroma.affymetrix/}].





%%%%%%%%%%%%%%%%%%%%%%%%%%%%%%%%%%%%%%%%%%%%%%%%%%%%%%%%%%%%%%%%%%%%%%%%%%%
% OTHER MODELS
%%%%%%%%%%%%%%%%%%%%%%%%%%%%%%%%%%%%%%%%%%%%%%%%%%%%%%%%%%%%%%%%%%%%%%%%%%%
\clearpage
\section{How raw copy numbers were estimated by other models}
\label{secOtherMethods}
In addition to CRMA~v2, two external methods were evaluated in this paper.  The first is Affymetrix' \emph{CN5} method~\citep{Affymetrix_2008m}, and the second is implemented in the dChip software~\citep{LiWong_2001}.

\subsection{CN5}
The CN5 method is implemented in the 'apt-copynumber-workflow' software part of the Affymetrix Power Tools (APT) v1.10.0.  The Affymetrix Genotyping Console (GTC)~v3.0 (build 3.0.3083.25494) software \citep{Affymetrix_2008m} utilizes APT for CN5 estimates.  We choose to run GTC, because it is not fully documented what settings should be used for APT.  According to Affymetrix both approaches produce identical results (Affymetrix Scientific Community Forums, Thread: 'copy number: Genotyping Console 3.0 vs. apt 1.10.0?' on August 15, 2008).
In CN5, probe signals are normalized ('adapter-type background correction') for systematic variation due to so called \emph{enzyme recognition-sequence class}.   Next, all probe signals (excluding control probes) are quantile normalized using the Affymetrix 'sketch' algorithm.  For SNPs, chip effects $\{(\theta_{Aij}, \theta_{Bij})\}$ (as in the log-additive model of RMA) are estimated separately  for the two alleles using the plier algorithm.  The total CNs are obtained by summing $\theta_{ij}=\theta_{Aij}+\theta_{Bij}$.
Log ratios are calculated as in Eqn~\eqref{CRMAv2:eqnCnLogRatio}, where the reference is $\theta_{Rj}=\median_i\{\theta_{ij}\}$ with the important difference that for ChrX (ChrY) it is only samples that empirically are found to females (males) that are included.  Finally, the raw CNs (log-ratios) are shifted such that the median of all median autosomal signals is zero.~\citep{Affymetrix_2008m}
There are some \emph{limitations/restrictions} in CN5 worth knowing about:
\begin{enumerate}
\item The CN5 method is available only for \GWSSix.  Affymetrix explicitly says that neither GTC nor APT implements CN5 for GWS5.
\item The CN5 method is limited to the default GWS6 CDF, that is, it cannot be used with the full GWS6 CDF.
\item The CN5 method use only females (males) when calculating reference on ChrX (ChrY).  In the current implementation of GTC is not possible to force CN5 to estimate raw CN ratios on ChrX (ChrY) using all samples.
\item The GTC software does not export $\{\theta_{ij}\}$ but only log-ratio CNs.
\end{enumerate}
It is because of the latter two restrictions we choose to calculate the CRMA~v2 and dChip estimates on ChrX and ChrY the same way as in CN5.  This is the only way a comparison of methods can be done.


%% From the SNPs and CN probes in the default CDF, GTC/CN5 filters out an additional set of 490 SNPs that are on ChrY (or without annotation) as well as 30,033 CN probes that was found to overlap ($\pm10$ base pairs from the center base) with known SNPs in dbSNP.


\subsection{dChip}
For the dChip model, we used the \pkg{dChip~2008} (Build: July 10, 2008, \url{http://www.dchip.org/}).  Probe-level data was normalized using the \emph{invariant-set method}~\citep{LiWong_2001}, and PM signals were background corrected by '5th percentile of region (PM-only)'.  For GWS6, array 'NA12750' was suggested by dChip to be used as the baseline array for normalization, because it had the median median (sic!) probe signal.  As suggested, we verified that the spatial intensity plot of this array was not abnormal.
For probe summarization, the dChip multiplicative model was used, with $\PM=\PMA+\PMB$ for SNPs (``Compute signals separately for A and B allele'' unchecked), returning MBEI scores (corresponding to $\{\theta_{ij}\}$).  For maximal comparison, the MBEI scores were imported to \pkg{aroma.affymetrix} and raw CNs where calculated as in Eqn~\eqref{CRMAv2:eqnCnLogRatio}.

%% We used GenomeWideSNP_6.Full.cdf
%% [GenomeWideSNP_6.Full.cdf.bin: 220,277,412 bytes]
%% To load all 60 arrays into memory in dChip takes roughly 1.4GB.  /HB 2008-08-19
%% In order to do PLM in dChip for GWS6, one has to set the memory to smallest possible, i.e. 250MB, which will make dChip allocate approx 1.6GB.  With 500MB (or 1000MB), dChip will try to allocate more than 2GB and because Windows won't give more than 2GB to a process, you will get an ``Out of Memory'' error dialog.  There is a ``/3GB'' Windows option I haven't considered. /HB 2008-08-19

\subsection{dChip*}
Due to odd performance of dChip for SNPs, we also ran the analysis where the MBEI probe summarization was replaced by averaging the signals while keeping everything else the same.  We denote this flavor of the dChip method by adding an asterisk to the label.


%% We are not aware of other freely available tools that can easily process GWS5 or GWS6 data for copy number analysis.

\begin{table}[hp]
\begin{center}
\begin{tabular}{|l|cc|c|c|cc|}
\hline
                             & CRMA~(v1) & CRMA~v2 & dChip & CNAG & CN4 & CN5 \\
\hline
\hline
Mapping10K\_Xba131           & yes     & yes       & yes   &    - &   - &  -  \\
Mapping10K\_Xba142           & yes     & yes       & yes   &    - &   - &  -  \\
\hline
Mapping50K\_Hind240          & yes     & yes       & yes   &  yes & yes &  -  \\
Mapping50K\_Xba240           & yes     & yes       & yes   &  yes & yes &  -  \\
\hline
Mapping250K\_Nsp             & yes     & yes       & yes   &  yes & yes &  -  \\
Mapping250K\_Sty             & yes     & yes       & yes   &  yes & yes &  -  \\
\hline
GenomeWideSNP\_5 (default)   &  -      & yes       & yes   &    - &  -  &  -  \\
GenomeWideSNP\_5 (full)      &  -      & yes       & yes   &    - &  -  &  -  \\
\hline
GenomeWideSNP\_6 (default)   &  -      & yes       & yes   &    - &  -  & yes \\
GenomeWideSNP\_6 (full)      &  -      & yes       & yes   &    - &  -  &  -  \\
\hline
Custom SNP \& CN chip types  &  yes    & yes       &   ?   &    - &  ?  &  ?  \\
\hline
\end{tabular}
\end{center}
\caption{Summary of methods that estimate raw CNs for the different Affymetrix SNP \& CN chip types.}
\label{tblSummaryOfMethods}
\end{table}



\clearpage
\section{Methods for the evaluation}
We base all the performance assessments using relative copy numbers (chip effects) on the non-logarithmic scale, that is, $C_{ij}=2\cdot\theta_{ij}/\theta_{Rj}$.  This is contrary to \cite{BengtssonH_etal_2008}, where we used log-ratios $M_{ij}=\log_2(\theta_{ij}/\theta_{Rj})$.  
%% If there are no negative $C_{ij}$, the full-resolution ROC estimates are identical, because the ROC is invariant under strictly increasing transforms.
We use ChrX and ChrY loci for the evaluation.  See Table~\ref{tblGenomicLocationByChromosome} for how many loci there are on each chromosome.
Loci in pseudo-autosomal regions (PARs) are excluded.  Each of the two sex-chromosomes have to PARs~\citep{BlaschkeRappold_2006}.  See Table~\ref{tblPARs} for details.
In addition to excluding PARs, regions known to be CN polymorphic \citep{RedonR_etal_2006} are excluded.  There are 48 such regions on ChrX and and 7 on ChrY.  We use a safety margin of 100kb on each side.
For further details on the evaluation methods are available in \cite{BengtssonH_etal_2008}.

\begin{table}[hp]
\begin{center}
\begin{tabular}{|r|c|c|c|cc|}
\hline
chromosome & PAR 1 & PAR 2 \\
\hline
\hline
X  & 1-2,692,881 & 154,494,747-154,824,264 \\
Y  & 1-2,692,881 &  57,372,174-57,701,691 \\
\hline
\end{tabular}
\end{center}
\caption{Pseudo-autosomal regions on ChrX and ChrY according to~\citet{BlaschkeRappold_2006}.  The regions are specified as base positions where the first position of the chromosome is index one.}
\label{tblPARs}
\end{table}




%% %%%%%%%%%%%%%%%%%%%%%%%%%%%%%%%%%%%%%%%%%%%%%%%%%%%%%%%%%%%%%%%%%%%%%%%%%%%
%% % Assessment of excluded loci
%% %%%%%%%%%%%%%%%%%%%%%%%%%%%%%%%%%%%%%%%%%%%%%%%%%%%%%%%%%%%%%%%%%%%%%%%%%%%
%% \section{Additional results}
%% 
%% \subsection{Assessment of excluded loci}
%% 
%% Since the full CDF contains additional SNPs that could potentially be used to estimate raw CNs, we wanted to investigate whether these loci can be used to identify single copies from diploids.  For GWS6, there are 1,115 (1.30\%) SNPs excluded from ChrX, which is representative to the amount excluded from an average chromosome (1.35\%) (Section 'Full and filtered sets of loci').
%% %% GWS5: 644 (2.39\%) ChrX SNPs excluded.
%% When filtering out loci in CN polymorphic and pseudo-autosomal regions, there are 919 ChrX loci left for evaluation on GWS6.  The ROC curves for the default and the excluded set of loci are shown in Figure~\ref{figROC-excluded}.  Although the excluded loci have lower TP rates, it is still clear that they have discriminatory power.
%% \begin{figure}[!tpbh]
%% \begin{center}
%%  \resizebox{0.38\columnwidth}{!}{\includegraphics{HapMap270,6_0,CEU,founders,plotROC,excludedLoci,all,y0=1_00-col}}%
%% \end{center}
%%  \caption{
%%   Performance of an average ChrX locus that has been excluded (919 loci) from the default (\GWSSix) CDF compared with the average ChrX locus in the default CDF (75,642 loci).  The ROC curves are based on raw-CN estimates from CRMA6.
%%  }
%%  \label{figROC-excluded}
%% \end{figure} 
%% 
%% This suggest that those loci could be included in downstream analysis in order to increase the effective resolution.  However, when identifying CN aberrations using segmentation methods, the assumption is that the raw CNs within a region have the same means.  When comparing density estimates of the raw CNs from the excluded loci with a same-size random set of default loci, we find, contrary to this general assumption, that there is a strong bias in the excluded loci.  This is likely the reason why Affymetrix excluded them (ref?).  The diploid loci have the same mean (0.00) and approximately the same standard deviation (0.20 v. 0.18), whereas the single-copy loci have a significantly higher mean for the excluded (-0.35) compared with the default (-0.65) set of loci.  This strong bias in the mean is likely to cause problems for a segmentation method.  
%% When calculating the TP rate as function of resolution using the method in \cite{BengtssonH_etal_2008}, which also relies on the assumption that neighboring loci should have the same mean level, the effective resolution does indeed decrease slightly when we include the excluded loci (not shown).
%% Furthermore, the excluded single-copy loci have a greater standard deviation (0.37) compared with the default loci (0.28) which explains the differences in ROC performances in Figure~\ref{figROC-excluded}.
%% \begin{figure}[!tpbh]
%% \begin{center}
%%  \resizebox{0.38\columnwidth}{!}{\includegraphics{HapMap270,6_0,CEU,founders,plotRawCNsDensity,excluded,all-col}}%
%%  \resizebox{0.38\columnwidth}{!}{\includegraphics{HapMap270,6_0,CEU,founders,plotRawCNsDensity,default,all-col}}%
%% \end{center}
%%  \caption{
%%   Empirical densities for the 30 males and 29 females of the raw CNs for the 919 excluded loci (left) and 919 random loci from the default \GWSSix CDF (right).  The diploid distributions are approximately the same for both sets, but for the single-copy loci the mean level (-0.35) of the excluded loci is substantially different from the default loci (-0.66).  The standard deviation is 0.37 and 0.26, respectively.
%%  }
%%  \label{figRawCN-excluded}
%% \end{figure}
%% Because we do not know the bias of these loci at other copy number levels, but also because we do not know their biases on the other chromosomes, it is not obvious how to correct for the biases.  An alternative to filtering out loci is to downweight them, according to some weight function, when fitting the segmentation models.  However, we do not know of any segmentation methods that take weights.  
%% 
%% 
%% \subsection{Assessment of GWS5 estimates}
%% In this section we give a brief summary of the results from estimating raw CNs in the GWS6 assay using the CRMA6, the APT, and the dChip method.  The GTC software does not support the GWS5 assay.
%% 
%% [Will run this when Affymetrix release the corrected annotation data...]

\clearpage
\bibliography{bioinformatics-journals-abbr,hb-at-maths.lth.se}
\bibliographystyle{natbib}
 

\end{document}
