\documentclass[12pt,a4paper]{article}
\usepackage{graphicx} % Note that 'graphics' is obsolete.
\usepackage{amsmath}
\usepackage{ifthen}
\usepackage{xspace}   % \xspace


% The search path (within curly brackets and separated by commas)
% where to find graphics files
\graphicspath{{figures/}}
%\DeclareGraphicsRule{.eps.gz}{eps}{.eps.bb}{`gunzip -c #1}

\newcommand{\ba}{\mathbf{a}}
\newcommand{\bb}{\mathbf{b}}
\newcommand{\by}{\mathbf{y}}
\newcommand{\package}[1]{\textbf{#1}\xspace}
\newcommand{\option}[1]{\textsl{#1}\xspace}
\newcommand{\name}[1]{\textsl{#1}\xspace}
\newcommand{\class}[1]{\textsl{#1}\xspace}

% More the LaTeX title page
\title{aroma.affymetrix - Developer's Guide}
\author{Henrik Bengtsson}
\date{August 11, 2006}

\begin{document}

\maketitle

\begin{abstract}
The purpose of the \package{aroma.affymetrix} package is to provide a memory efficient, user and developer friendly, and simple interface to read, write and model Affymetrix data from any chip type.\\

It provides the foundation to do higher-level analysis, but it does \emph{not} implement chip-type specific analysis.  Such methods are supposed to be implemented in other packages, e.g. \package{aroma.affymetrix.snp} and \package{aroma.affymetrix.expression}.\\

This document describes the design of the package.
\end{abstract}


\section{Introduction}
An Affymetrix CEL file contains information about the probes but no details about the layout of such.  The layout and what probes belong to what probesets are defined in the CDF file.  Typically, a CEL file contains information of the probes coordinate on the array (\name{x} and \name{y}), the average intensity (\name{intensities}) and the standard deviation (\name{stdvs}) and the number (\name{pixels}) of the pixels within each probe.  We write typically, because in the new Calvin binary file format there is an option to store only a subset of these fields.  However, for the ASCII (v3) and binary (v4) CEL file format version, the above fields are mandatory.

In order to access, but also update, the above fields in CEL files (and other formats), the \package{affxparser} package was developed.  That package is utilizing the Fusion SDK I/O library (open-source, LGPL, C++) of Affymetrix.  The advantages of using Fusion SDK are many.  Since we use the same code base as other developers (and users), many more CPU hours are used to test the code, increasing the chances to find bugs and hence improve the overall quality of Fusion SDK, but also our package.  We also do not have to worry about new versions of different file formats, because support for these will be implemented by Affymetrix and all we have to do is write a wrapper, or not even that.  This was for instance the case when the Calvin file format was released.  The \package{affxparser} package is fast, it allows one to read subsets of the data at the probe level but also at the probeset level, even across multiple arrays.

The \package{aroma.affymetrix} package defines a set of classes to access and manipulate Affymetrix data on a higher level that what is provided by the \package{affxparser} package.  For instance, it provides basic methods for normalizing data, and ways to apply summarization methods on one or several probesets.  
The number one design strategy of the package is that data is living on the file system, and not in memory.  This will allow any analysis method to scale with the number of arrays in the dataset.

\section{Classes}
In \package{aroma.affymetrix} there are two classes that are more important to know of than any other classes, namely the \class{AffymetrixDataFile} and \class{AffymetrixDataSet} class.  
The \class{AffymetrixDataFile} class represents a \emph{single} Affymetrix data file, e.g. a CEL file.
The \class{AffymetrixDataSet} class represents a \emph{set of} Affymetrix data files, i.e. a set of \class{AffymetrixDataFile} objects.

The above two classes are both defined to be abstract, that is, they can not be instantiated as is.  The \class{AffymetrixCelFile} class, which is a subclass of \class{AffymetrixDataFile}, represents a CEL data file.  This is most likely the class you are going to work with.

\subsection{AffymetrixCelFile}
Given an \class{AffymetrixCelFile} object, you can query in different ways.  For instance, you may get the \emph{pathname} to the CEL file it represents, get and set the \emph{probe intensities} and so on.  Since there is a CDF file associated with each CEL file, you can also get an \class{AffymetrixCdfFile} object representing the CDF file.  

\subsection{AffymetrixCdfFile}
Given an \class{AffymetrixCdfFile} object, you can get information on the \emph{chip type}, the \emph{number of probes and probesets} on the chip, the \emph{probeset names} and so on.

\subsection{AffymetrixCelDataSet}
The \class{AffymetrixCelDataSet} class is a subclass of the \class{AffymetrixDataSet} and represents a set of \class{AffymetrixCelFile} objects (which all must be of the same chip type).  You can quiry it for the \emph{number of arrays} in the data set, the \class{AffymetrixCdfFile} object, and so on.


\section{Caching}
The \package{aroma.affymetrix} package provides a caching mechanism at the probe level.  This means that when the same set of probes are queried multiple times, the data does not have to be re-read from the data files.

The caching mechanism is totally transparent for the end user.  The only thing that user might have to know about is how to set the size of the cache, which is done via the \option{aroma.affymetrix/cacheSize} option.

As a developer it good to understand how the caching mechanism works in order to optimize cache-hit ratio, but otherwise everything else is taken care of by \package{aroma.affymetrix}.
\end{document}

